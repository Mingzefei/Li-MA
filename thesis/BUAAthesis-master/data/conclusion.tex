% !Mode:: "TeX:UTF-8"
\chapter*{结论\markboth{结论}{}}
\addcontentsline{toc}{chapter}{结论}

本课题在Yao等人\upcite{yinMetalChloridePerovskite2020}的工作基础之上,针对锂离子在 \ce{MASnCl3}类有机金属钙钛矿中的迁移过程进行深入研究。
主要通过计算模拟的方法,建立了\ce{MASnCl3}、\ce{MASnBr3}和\ce{MAPbI3}等体系的锂离子嵌入模型和锂离子迁移模型,分析迁移机制,取得的主要结论如下:

\begin{enumerate}
    \item 该类钙钛矿中甲胺离子的胺基一端更倾向于靠近无机框架中的卤原子,形成更高强度的氢键;
    \item 锂离子嵌入体系时,有机分子为减少嵌入导致的位阻会改变取向,并且由于氢键强度的差异,有机分子的胺基端远离锂离子,相应的甲基端接近锂原子,进而产生较弱的相互作用;
    \item 锂离子在该类钙钛矿中具有较低的迁移能垒(\SI{0.2}{eV}至\SI{0.5}{eV}),且具有各向异性,在与甲胺离子原本取向相近的方向上能垒相对较低(\ce{MASnCl3}体系中,最低为 \SI{0.29}{eV});
    \item 甲胺离子随锂离子迁移而发生旋转,以较低的能量代价减小锂离子位阻,该旋转机制是迁移能垒较低和各向异性的原因;
    \item 锂离子在迁移过程中,与周围3至5个卤原子形成离子键,导致无机八面体发生畸变,但无机框架依然维持稳定。 
\end{enumerate}

