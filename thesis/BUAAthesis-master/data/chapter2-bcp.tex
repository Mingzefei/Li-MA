% !Mode:: "TeX:UTF-8"
\chapter{有机金属卤化物钙钛矿中氢键的分析方法}

\section{引言}

在有机金属卤化物钙钛矿系统中,有机分子的氢原子和无机框架中的卤素原子容易形成氢键,进而影响体系的结构、性能。为此,人们发挥计算模拟的优势,针对氢键强度,及其对整体结构的影响,进行研究。
例如,Svane等人\upcite{svaneHowStrongHydrogen2017}开发出计算该体系氢键强度的方法,并发现该类化合物的氢键键能明显弱于典型的分子氢键;
Varadwaj等人\upcite{varadwajSignificanceHydrogenBonding2019}深入研究了氢键和其他非共价相互作用对无机八面体倾斜的影响。

对于本文研究的体系,大量存在的氢键可能对锂离子的迁移路径及能垒产生不可忽略的影响。
根据文献\cite{svaneHowStrongHydrogen2017},有机金属卤化物钙钛矿中氢键键能为\SI{0.09}{eV}至 \SI{0.26}{eV},与文献\cite{yinMetalChloridePerovskite2020}初步计算的锂离子迁移能垒 \SI{0.45}{eV} 相当。
因此,为研究氢键对锂离子迁移过程中能垒的影响,有必要开发出利用 VASP 计算结果快速获取体系氢键键能等性质的方法。

\section{利用分子中的原子理论研究氢键}

分子中的原子理论\upcite{baderQuantumTheoryMolecular1991b}(Atoms in molecules,AIM)是一种量子化学模型,基于电子密度标量场的拓扑性质划分空间,使得每个区域仅包含一个原子核,从而定义出量子化学观点下的原子,并给出原子的部分性质。
该理论在提出之后不断发展,广泛应用于化学领域理论研究,特别是原子间的成键分析。

根据AIM理论,当且仅当两个原子之间被一个零通量面分隔开,且该零通量面上有一个$(3,-1)$临界点时,认为两个原子间存在键。
其中临界点指电子密度梯度为零的点,$(3,-1)$临界点指该临界点的海森矩阵的本征值中有两个负值和一个正值,该点被称为键临界点(bond critical point,bcp)。
几何意义上,键临界点为电子密度标量场上的一阶鞍点;代数意义上,键临界点为满足如下关系式的空间位置:

\begin{gather}
    \Vert \nabla\rho(x,y,z)  \Vert = 0; \\
    (\nabla\nabla\rho)u_i = \lambda_i u_i\quad (i=1,2,3); \\
    \lambda_i \neq 0 \quad (i=1,2,3); \\
    \sum_{i=1}^3\text{sign}(\lambda_i) = -1.  
\end{gather}
其中,
\begin{gather}
    \nabla\rho(x,y,z) = 
    \begin{bmatrix}
        \partial\rho/\partial x & \partial\rho/\partial y & \partial\rho/\partial z
    \end{bmatrix}; \\
    \nabla\nabla\rho= 
    \begin{bmatrix}
        \displaystyle\frac{\partial^2 \rho}{\partial x^2} & \displaystyle\frac{\partial^2 \rho}{\partial x\partial y} & \displaystyle\frac{\partial^2 \rho}{\partial x \partial z} \\
        \displaystyle\frac{\partial^2 \rho }{\partial y \partial x} & \displaystyle\frac{\partial^2 \rho }{\partial y^2} & \displaystyle\frac{\partial^2 \rho }{\partial y \partial z} \\
        \displaystyle\frac{\partial^2 \rho }{\partial z \partial x} & \displaystyle\frac{\partial^2 \rho }{\partial z \partial y} & \displaystyle\frac{\partial^2 \rho }{\partial z^2}
    \end{bmatrix}.
\end{gather}

一般而言,同类型键的bcp处电子密度与键的强度呈正相关。
在最近的研究中,Emamian等人\upcite{emamianExploringNaturePredicting2019}通过构造一系列具有代表性的氢键体系,通过量子化学的方法计算证明,利用氢键的键临界点处的电子密度可以可靠地估计氢键强度,并给出了相应的关系式,如图\ref{fig:bcp-energy}所示。

\begin{figure}[htbp]
    \centering
    \includegraphics[width=1\textwidth]{bcp-energy.png}
    \caption{(a)所有计算样本;(b)电中性计算样本;(c)非电中性计算样本。}
    \label{fig:bcp-energy}
\end{figure}

图\ref{fig:bcp-energy}(a)为针对所有氢键体系的bcp处电子密度和氢键键能直接拟合的结果,尽管决定系数 $R^2$ 较好,但平均绝对百分误差 MAPE 很大,因此拟合结果在预测氢键键能上无实际意义;图\ref{fig:bcp-energy}(b)和(c)为针对不带电氢键体系和带电体系的bcp处电子密度和氢键键能分别进行拟合的结果,决定系数 $R^2$ 和 平均绝对百分误差 MAPE 结果均很好。

但是该工作主要基于量子化学方法和分子体系,尚无充分的数bb证明关系式适用于DFT理论下的固体体系。
本课题将对此进行验证。

\section{代码实现}

该部分将说明利用VASP计算结果获取bcp处电子密度的方法。

VASP 的输出文件 CHGCAR 中记录了体系的原子核位置和价电子空间电子密度。
为保证结果的准确性,通过在 VASP 的计算设置文件 INCAR 中增加 “LAECHG=.TRUE.” 并通过 VTST 提供的脚本获得包含体系全部电子的空间电子密度文件 CHGCAR\_sum。
针对该文件编写 python 脚本,获取体系的 bcp 处电子密度。
其伪代码如下,完整代码已上传至 GitHub\footnote{https://github.com/Mingzefei/Li-MA/blob/master/relax/AIM\_bcp.py}。


\begin{algorithm}
	\renewcommand{\algorithmicrequire}{\textbf{Input:}}
	\renewcommand{\algorithmicensure}{\textbf{Output:}}

    \caption{获取指定原子间 bcp 处电子密度}
	\label{alg1}
	\begin{algorithmic}[1]
                \Require 
                        $atomA(x^a,y^a,z^a)$: 原子A坐标;
                        $atomB(x^b,y^b,z^b)$: 原子B坐标;
                        $\rho(x_n,y_n,z_n)$: 离散空间电子密度;
                        $eps1,eps2$: 条件阈值
                \State 转换原子A和B实际坐标到离散空间中坐标 $(x_n^a,y_n^a,z_n^a),(x_n^b,y_n^b,z_n^b)$
                \State $ grad(x_n,y_n,z_n) \gets \nabla \rho(x_n,y_n,z_n)$
                \State $ hess(x_n,y_n,z_n) \gets \nabla grad(x_n,y_n,z_n)$
                \State $ chg^{bcp} \gets 0$
                \For{ $ x_n $ in $ \mathrm{range}(\min(x_n^a,x_n^b),\max(x_n^a,x_n^b)) $ }
                \For{ $ y_n $ in $ \mathrm{range}(\min(y_n^a,y_n^b),\max(y_n^a,y_n^b)) $ }
                \For{ $ z_n $ in $ \mathrm{range}(\min(z_n^a,z_n^b),\max(z_n^a,z_n^b)) $ }
                        \State $ module \gets \Vert grad(x_n,y_n,z_n) \Vert$
                        \If {$ module < eps1$}
                                \State $\lambda_1,\lambda_2,\lambda_3 \gets \mathrm{eig}(hess(x_n,y_n,z_n)) $
                                \State $condition1 \gets \bigwedge_{i=1}^3 \left[\mathrm{abs}(\lambda_i)< eps2\right]$
                                \State $condition2 \gets \left[\sum_{i=1}^3 \mathrm{sign}(\lambda_i)\right] = -1$
                                \If{$condition1 \wedge condition2 $}
                                        \If{$chg^{bcp} = 0 \vee module^{bcp}>module$}
                                                \State $chg^{bcp} \gets chg(x_n,y_n,z_n)$
                                                \State $module^{bcp} \gets module $
                                        \EndIf
                                \EndIf
                        \EndIf
                \EndFor
                \EndFor
                \EndFor
                \Ensure $chg^{bcp}$
	\end{algorithmic} 
\end{algorithm}

\section{计算结果及分析}

这部分通过构建相关体系,验证上述方法的有效性,即通过 VASP 可以较准确获取bcp处电子密度,以及通过bcp处电子密度可以对课题体系的氢键强度进行估计。

\subsection{VASP计算结果的准确性}

文献\cite{emamianExploringNaturePredicting2019}中使用量子化学方法进行计算,而VASP软件通过DFT方法求解能带进行计算,两者计算方法存在差异。因此,有必要验证两种方法对于bcp处电子密度计算的一致性。

考虑到本课题研究的体系中多为 $\mathrm{N}-\mathrm{H}\cdots\mathrm{X}$(其中$\mathrm{X}=\mathrm{Cl,Br,I}$)类氢键,这里选用文献\cite{emamianExploringNaturePredicting2019}中 $\rm FH \cdots NH_3$(28号)、$\rm H_2NH \cdots F^-$(33号)和$\rm HF\cdots HNH_3^+$(30号)分子体系进行建模和VASP 计算,并求解bcp处电子密度。
建立 \SI{15x15x15}{\angstrom} 晶胞,使用The Materials Project数据库\footnote{https://materialsproject.org}中分子数据,其余参数使用软件建议值或默认值。最终结果如图\ref{fig:bcp-test1}。

\begin{figure}[htbp]
    \centering
    \includegraphics[width=0.75\textwidth]{bcp-test1.png}
    \caption{部分氢键体系bcp处电子密度的VASP计算值与文献值}
    \label{fig:bcp-test1}
\end{figure}

根据计算结果,利用VASP进行的DFT计算获得的bcp处电子密度较文献\cite{emamianExploringNaturePredicting2019}中使用的量子化学方法存在正向偏差,偏高 \SI{13}{\percent} 至 \SI{21}{\percent}。
该结果推测是由计算方法导致。一般而言,量子化学方法对分子体系的计算更为准确,而利用VASP进行的DFT计算,由于波函数展开方式、赝势等原因,对分子体系的计算更易产生偏差。
当前误差对于本课题仍可接受,可用于成键强度的定性比较。

\subsection{相关参数的精度检验}

VASP对体系电子密度的计算结果,除受原子位置、赝势、波函数展开方式等影响之外,也可能受截断能、空间网格划分等计算参数影响。
该部分将通过一系列实验,判断K点数量、截断能、空间网格划分(对应于KPOINTS输入文件,以及INCAR输入文件中ENCUT和NGXF、NGYF、NGZF参数)等对计算精度的影响,并由此确定合适的计算参数。

测试使用本课题研究的 \ce{Li+}嵌入\ce{MASnCl3} 体系,以其中的 锂离子和21号氢原子间 bcp 电子密度为测试对象,分别设置KPOINTS为 $2 \times 2 \times 2$、$3 \times 3 \times 3$、$4 \times 4 \times 4$、$5 \times 5 \times 5$、$6 \times 6 \times 6$;ENCUT为 $450$、$500$、$550$、$600$、$650$;NGF(NGXF、NGYF、NGZF参数均与NGF相等)为$150$、$168$、$192$、$200$、$224$。计算结果如图\ref{fig:bcp-test2}。

\begin{figure}[htbp]
    \centering
    \subfigure{
        \includegraphics[width=0.3\textwidth]{kpoint-test.png}
    }
    \subfigure{
        \includegraphics[width=0.3\textwidth]{encut-test.png}
    }
    \subfigure{
        \includegraphics[width=0.3\textwidth]{ngf-test.png}
    }
    \caption{VASP的计算参数(a)KPOINTS;(b)ENCUT和(c)NGF对bcp处电子密度计算精度的测试}
    \label{fig:bcp-test2}
\end{figure}


由结果可以看出,ENCUT和NGF参数对bcp电子密度影响最为显著(计算结果与软件默认参数所得结果相差约 \SI{15}{\percent}),而KPOINTS对计算结果几乎无影响。

这里综合考虑计算精度和计算成本,将KPOINTS设置为$4 \times 4 \times 4$;ENCUT设置为$600$;NGF设置为$192$,进行VASP的电子自洽计算,进而求解bcp处电子密度。

\subsection{结果的有效性}

文献\cite{svaneHowStrongHydrogen2017}通过分割有机阳离子和无机框架的方法,计算了 \ce{MAPbI3} 等体系中的氢键键能,为 \SI{0.09}{eV}至 \SI{0.27}{eV}平均每个阳离子(\ce{MAPbI3}体系为\SI{0.27}{eV}平均每个阳离子,\SI{0.09}{eV}平均每个氢键)。
本课题针对 \ce{MAPbI3} 体系建模计算,发现其中的有机阳离子提供的氢键配体数量更为复杂,而非文献\cite{svaneHowStrongHydrogen2017}中给出的固定数量3;
但是,通过将VASP求解得到的 bcp处电子密度 \SI{0.012}{a.u.} 至 \SI{0.017}{a.u.}代入文献\cite{emamianExploringNaturePredicting2019}所提供的电中性体系氢键键能拟合关系式,可得相应的单个氢键键能为\SI{0.08}{eV}至 \SI{0.13}{eV},与文献值\cite{svaneHowStrongHydrogen2017}\SI{0.09}{eV}相比略有高估,这与上文 VASP 高估 bcp 处电子密度的结论相一致。

\section{结论}

该部分实现并验证了根据VASP计算结果求解AIM理论中bcp处电子密度的方法,
发现使用VASP的DFT计算较量子化学方法会高估bcp处电子密度;
并将该方法应用于有机金属卤化物钙钛矿体系的氢键强度估计,
通过与文献值比较,认为该方法可以简单、半定量地获得体系的氢键强度。
