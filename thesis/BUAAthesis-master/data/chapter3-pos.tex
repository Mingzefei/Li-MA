% !Mode:: "TeX:UTF-8"
\chapter{锂离子嵌入有机金属卤化物钙钛矿的计算}

\section{引言}

有必要对锂离子嵌入有机金属卤化物钙钛矿进行计算,以获取热力学上相对稳定的嵌入结构:
一方面,稳定的嵌入结构是后续利用CI-NEB方法模拟锂离子输运过程的基础;
另一方面,稳定的嵌入结构在更精确的电子自洽计算后,有助于揭示材料的电子结构等信息。

该部分将首先对锂离子嵌入模型建模并优化结构,获得热力学上稳定构型;
之后提高计算精度进行电子自洽计算,获得电子结构等信息;
最后利用前文AIM等理论,分析计算结果。

\section{模型搭建}

\subsection{钙钛矿的理想结构模型}

根据绪论章节对有机金属卤化物钙钛矿的介绍,该类钙钛矿在不同温度下存在不同晶格类型,且其中的有机分子也会导致无机八面体的倾斜。
当温度为 \SI{0}{K} 时,有机分子的振动、旋转等影响可被忽略,并依据能量最低,按相同取向填入无机框架。
此时有机金属卤化物钙钛矿整体呈正交相。
可以称该结构为理想结构。

利用 The Materials Project数据库\footnote{https://materialsproject.org}及文献\cite{yinMetalChloridePerovskite2020}表征的晶格参数等数据,搭建 \ce{MASnCl3}、\ce{MASnBr3}、\ce{MASnI3}和 \ce{MAPbI3} (\ce{MA}为甲胺离子\ce{CH3NH3+})四种体系的结构模型。
为之后研究锂离子嵌入和迁移时降低锂离子间影响,对钙钛矿模型进一步扩胞处理,形成$2\times2\times2$的超胞;
降低晶体结构的对称性,将对称群设置为 $P1$。
如图\ref{fig:ideal-struc}所示,结构优化后结果将在下节给出。

\begin{figure}[ht]
    \centering
    \includegraphics[width=0.75\textwidth]{../data/111-ideal/better-CONTCAR.png}
    \caption{扩胞处理后的钙钛矿理想结构(绿色球为卤族原子,深灰色球为 \ce{Sn}或 \ce{Pb},浅灰色、棕色和粉色球分别为 \ce{N}、\ce{C}和 \ce{H})}
    \label{fig:ideal-struc}
\end{figure}

\subsection{锂离子嵌入位置}

受位阻影响,锂离子嵌入钙钛矿的八面体空隙中心时能量最低。
结合钙钛矿的结构(图\ref{fig:perovskite}),八面体空隙中心均位于其面心。
考虑到有机分子 \ce{CH3NH3+} 具有取向,对于单个有机金属钙钛矿晶胞而言,其三个面心(由于一个面心共用于两个晶胞,实际仅有三个面心)互不等价,即无法通过立方体的旋转操作使不同面心重合。
这意味着,三个面心具有不同的化学环境。
将这三个面心分别记为 pos1、pos2和pos3。

\begin{figure}[ht]
    \centering
    \includegraphics[width=0.75\textwidth]{../data/111-ideal/3pos.png}
    \caption{三个不等价的面心(红球所示)}
    \label{fig:3pos}
\end{figure}


因此,针对上述不同位置分别插入锂原子,并控制系统总电子数模拟锂离子所带正电荷,固定晶格参数,如图\ref{fig:pos1-poscar}、图\ref{fig:pos2-poscar}和图\ref{fig:pos3-poscar}所示,结构优化后结果将在下节给出。

\begin{figure}[ht]
    \centering
    \includegraphics[width=0.75\textwidth]{../data/111-11-pos1/POSCAR.png}
    \caption{pos1处嵌入锂离子(红色球为锂离子)}
    \label{fig:pos1-poscar}
\end{figure}
\begin{figure}[ht]
    \centering
    \includegraphics[width=0.75\textwidth]{../data/111-21-pos2/POSCAR.png}
    \caption{pos2处嵌入锂离子(红色球为锂离子)}
    \label{fig:pos2-poscar}
\end{figure}
\begin{figure}[ht]
    \centering
    \includegraphics[width=0.75\textwidth]{../data/111-31-pos3/POSCAR.png}
    \caption{pos3处嵌入锂离子(红色球为锂离子)}
    \label{fig:pos3-poscar}
\end{figure}



\section{计算结果分析}

\subsection{计算参数}

该部分计算包括两部分:结构优化计算和电子自洽计算。
两者的计算目的和计算代价不同,需要设置相应的计算参数。

计算结构优化时,截断能ENCUT依据软件建议值设置为500,电子步收敛判据EDIFF设置为1E-5,离子步收敛判据EDIFG设置为-0.01;KPOINTS设置为以gamma点为中心自动生成$3\times3\times3$网格。

计算电子自洽时,综合前文对bcp处电子密度的测试,截断能ENCUT提高至600,电子步收敛判据EDIFF提高至1E-6;KPOINTS网格增加至$4\times4\times4$,计算电子密度时网格数量NGXF、NGYF和NGZF均设置为196。

此外,由于计算体系为半导体,波函数展开方法ISMEAR设置为0(高斯展宽),展宽SIGMA设置为0.10;
由于计算体系中存在分子,需考虑范德华作用,IVDW设置为11,使用DFT-D3方法进行范德华修正;
为模拟锂离子所带的一个单位正电荷,在电中性体系价电子数量为403的基础上减少一个电子,NELECT设置为402。
使用GGA-PBE泛函。

\subsection{稳定结构}

通过结构优化,获得\ce{MASnCl3}、\ce{MASnBr3}、\ce{MASnI3}和 \ce{MAPbI3}四种体系的理想结构和三种锂离子嵌入结构。

由于四种体系的计算结果类似,仅以图片形式展示\ce{MASnCl3}的理想结构(图\ref{fig:111-ideal-contcar})和三种锂离子嵌入结构(图\ref{fig:111-11-pos1-contcar}、\ref{fig:111-21-pos2-contcar}和\ref{fig:111-31-pos3-contcar})。

\begin{figure}[ht]
    \centering
    \includegraphics[width=0.75\textwidth]{../data/111-ideal/CONTCAR.png}
    \caption{结构优化后的\ce{MASnCl3}理想结构}
    \label{fig:111-ideal-contcar}
\end{figure}
\begin{figure}[ht]
    \centering
    \includegraphics[width=0.75\textwidth]{../data/111-11-pos1/CONTCAR.png}
    \caption{结构优化后的\ce{Li+}嵌入\ce{MASnCl3}pos1 位置结构}
    \label{fig:111-11-pos1-contcar}
\end{figure}
\begin{figure}[ht]
    \centering
    \includegraphics[width=0.75\textwidth]{../data/111-21-pos2/CONTCAR.png}
    \caption{结构优化后的\ce{Li+}嵌入\ce{MASnCl3}pos2 位置结构}
    \label{fig:111-21-pos2-contcar}
\end{figure}
\begin{figure}[ht]
    \centering
    \includegraphics[width=0.75\textwidth]{../data/111-31-pos3/CONTCAR.png}
    \caption{结构优化后的\ce{Li+}嵌入\ce{MASnCl3}pos3 位置结构}
    \label{fig:111-31-pos3-contcar}
\end{figure}

\begin{table}
    \centering
    \caption{不同钙钛矿体系理想结构的优化后晶格参数}
    \label{tb:cell-paramter}
    \begin{tabular}{ccccccc}
        \toprule
        钙钛矿体系 & a/\si{\angstrom} & b/\si{\angstrom} & c/\si{\angstrom} & $\alpha$/\si{\degree} & $\beta$/\si{\degree} & $\gamma$/\si{\degree}\\
        \midrule
        \ce{MASnCl3}    & 11.11568 & 11.43881 & 11.52084 & 86.73 & 89.60 & 90.29 \\
        \ce{MASnBr3}    & 11.62841 & 11.80050 & 11.91464 & 86.08 & 89.73 & 90.39 \\
        \ce{MASnI3}    & 12.38372 & 12.54657 & 12.58601 & 87.35 & 89.72 & 90.29 \\
        \ce{MAPbI3}    & 12.57420 & 12.69275 & 12.78958 & 88.80 & 89.95 & 89.97 \\           
        \bottomrule
    \end{tabular}
\end{table}

根据计算结果,结构优化后的理想结构晶格参数如表\ref{tb:cell-paramter}所示。
比较不同体系的晶格参数可知,随着构成无机框架原子的原子半径增大,晶格参数也随之增大。

\begin{table}
    \centering
    \caption{不同钙钛矿体系理想结构的\ce{H}与临近卤原子距离}
    \label{tb:HX-dist}
    \begin{tabular}{ccc}
        \toprule
        钙钛矿体系 & \ce{NH3}端H与临近卤原子距离/\si{\angstrom} & \ce{CH3}端H与临近卤族原子距离/\si{\angstrom}\\
        \midrule
        \ce{MASnCl3}    & 2.25, 2.51 & 2.90 \\
        \ce{MASnBr3}    & 2.43, 2.79& 2.95 \\
        \ce{MASnI3}    & 2.67, 2.91 & 3.27 \\
        \ce{MAPbI3}    & 2.69, 2.80 & 3.38  \\              
        \bottomrule
    \end{tabular}
\end{table}

钙钛矿中的有机分子,即甲胺离子,值得注意。

在理想结构中,甲胺离子的胺基一端更倾向于靠近无机框架中的卤族原子(表\ref{tb:HX-dist})。
当无机框架尺寸较大时,这一现象更为明显。

当锂离子嵌入钙钛矿体系时,其中的有机分子取向发生明显改变。
整体而言,甲胺分子的甲基一端更接近锂离子,而胺基端更接近无机框架中的卤原子。
对于不同的嵌入位置,甲胺分子取向的变化程度也不同。
pos2位置的“头碰头”形式,较pos1和pos3的“肩并肩”形式,取向变化更为明显。

上述现象将在下文电子结构中进一步分析。

\subsection{电子结构}

DOS 

bcp

\section{结论}

不同位置

不同结构

