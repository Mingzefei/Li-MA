% !Mode:: "TeX:UTF-8"
\chapter{绪论}
\section{锂金属电极保护膜及离子输运}

锂离子电池以其高工作电压、大容量等优点,在电化学储能领域引起人们极大的兴趣。
但是,由于锂金属独特的活泼型,
锂离子电池在充放电循环中极易发生金属电极体积膨胀、锂枝晶生长等现象,使得电池破坏、短路的风险大大提高;
金属电极与电解液之间复杂的界面反应,以及循环中形成的“死锂”(图\ref{fig:Li-anode})会导致电池容量下降,降低电池效率。
这些都阻碍了锂金属电极的应用范围。

\begin{figure}[htbp]
    \centering
    \subfigure[锂枝晶和死锂]{
    \includegraphics[width=0.23\textwidth]{Li-andoe-a.jpg}
    }
    \subfigure[抑制枝晶的方法]{
    \includegraphics[width=0.71\textwidth]{Li-andoe-b.jpg}
    }
    \caption{锂金属电极在充放电过程中产生锂枝晶和死锂及抑制锂枝晶的方法\upcite{huaApplicationComputationalSimulation2020}}
    \label{fig:Li-anode}
\end{figure}

目前,人们从实验上证明,设计合适的保护层是解决上述问题的有效手段之一。
已开发的保护层包括聚合物型、无机型,以及合金型等,利用不同机制发抑制锂枝晶的形成和生长。
例如, Lopez等人\upcite{lopezEffectsPolymerCoatings2018}对聚合物型保护层进行分析和比较,发现该类保护层的机械性能影响锂离子沉积的宏观均匀性,而其化学性质则是影响局域沉积成核生长的主要原因。
Liu等人\upcite{liuMixedLithiumIon2020}通过气固反应制备出\ce{Li2S/Li2Se}保护层,使得\ce{LiFePO4}及\ce{LiNi_{0.6}Co_{0.2}Mn_{0.2}O2}全电池在电池循环性能上均有所提升。其中,\ce{Li2Se}较\ce{Li2S}的锂离子迁移能垒更低,表现出更好的离子导电率,在一定程度上抑制了锂枝晶的形成。
Guo等人\upcite{guoDendritefreeLithiumDeposition2020}根据\ce{Ag}和\ce{Au}的良好亲锂性,通过在电极表面构建具有低锂离子成核势垒的合金层,改善锂的沉积均匀性。

保护层对锂金属电极的保护作用主要包括两方面:隔绝活泼的锂金属和电解液,减少两者之间复杂的界面反应,从而提高电池效率;调控锂离子的迁移和沉积过程,抑制锂枝晶的生长,从而提高电池安全性\upcite{huaApplicationComputationalSimulation2020}。
因此,为有针对性地设计新型高效保护膜,锂离子在不同类型保护膜中的输运行为引起了人们的注意。
Wei等人\upcite{weiKineticsTuningLiIon2015}通过理论计算和实验相结合,阐明了\ce{LiNi_{x}Mn_{y}Co_{z}O2}($x+y+z=1$)材料中调节锂离子扩散动力学的方法。随着材料体系含锂量的不同,存在两种不同的迁移方式。
Dathar等人\upcite{datharCalculationsLiIonDiffusion2011}较为系统地研究了橄榄石磷酸盐\ce{FePO4}和\ce{LiFePO4}中的锂离子扩散动力学,通过考虑块体、表面、缺陷等因素影响,解释了前人理论计算和实验测量间的数值差距。
Yang等人\upcite{yangLiIonDiffusion2011}从实验表征和理论计算上对橄榄石磷酸盐\ce{LiFePO4}中锂离子迁移机理进行研究,发现其中包括两个过程:锂离子在\ce{PO4}基团相邻\ce{Li}位点之间跳跃;\ce{Fe}离子协同运动,进而形成反位缺陷,促进锂离子扩散。
Iddir等人\upcite{iddirLiIonDiffusion2010}对单斜\ce{Li2CO3}中锂离子迁移进行计算模拟,发现锂离子沿\ce{Li2CO3}开放通道的能垒(\SI{0.28} {eV})远低于跨平面方向能垒(\SI{0.60}{eV}),该过程中 \ce{Li+} 与\ce{O}依次发生断键和成键。
在近期的研究中Yao等人\upcite{yinMetalChloridePerovskite2020}将光电材料有机金属卤化物钙钛矿用作锂金属电极保护膜,实验结果证明了其有效性。而有机金属卤化物钙钛矿为锂离子提供了新型的输运环境,其中存在的复杂作用关系,可能启发人们设计新型保护膜。

综上,为开发更有效的锂金属电极保护膜,提高锂电极性能,人们已经对 \ce{Li2CO3} 、\ce{LiFePO4} 等锂电池体系材料的锂离子输运行为进行多方面的研究,并发现其中的机理受环境、缺陷等因素影响,该过程中锂离子与相关原子的成键断键或协同运动都会显著影响迁移能垒。

\section{有机金属卤化物钙钛矿}

有机金属卤化物钙钛矿因其合适的能带结构,被用于研发高效太阳能电池\upcite{liRecentProgressOrganohalide2015}。
当前,对于影响有机金属卤化物钙钛矿光电性能的晶体结构、掺杂特性等因素,人们已有较全面的研究。

\begin{figure}[htbp]
    \centering
    \includegraphics[width=0.3\textwidth]{perovskite.png}
    \caption{有机金属卤化物钙钛矿的结构示意图(该图片来源于网络)}
    \label{fig:perovskite}
\end{figure}

钙钛矿的基本结构如图\ref{fig:perovskite}所示,由B位金属离子和X位卤素原子形成无机框架,A位阳离子填充其中。

有机金属卤化物钙钛矿的晶格类型受温度影响很大。
以 \ce{CH3NH3PbI3}(如图\ref{fig:temp-struc}所示)为例,不同温度下,晶格中的\ce{PbI3}八面体发生不同程度的旋转和倾斜,从而产生不同的相:
低温时为正交相;温度在\SI{162}{K}至\SI{330}{K} 时为四方相;高温 时为立方相\upcite{liRecentProgressOrganohalide2015}。

\begin{figure}[htbp]
    \centering
    \subfigure[正交相]{
    \includegraphics[width=0.3\textwidth]{temp-struc-a.png}
    }
    \subfigure[四方相]{
    \includegraphics[width=0.3\textwidth]{temp-struc-b.png}
    }
    \subfigure[立方相]{
    \includegraphics[width=0.3\textwidth]{temp-struc-c.png}
    }
    \caption{不同温度下碘铅甲胺钙钛矿的晶体结构(框架中黑球、灰球分别为\ce{Pb}和\ce{I},有机分子中灰色、浅灰色和白球分别为\ce{N}、\ce{C}和\ce{H})\upcite{liRecentProgressOrganohalide2015}}
    \label{fig:temp-struc}
\end{figure}

对于该类钙钛矿中的有机阳离子A,如\ce{MA}(\ce{CH3NH3},甲胺),由于 \ce{C}和 \ce{N}电负性不同,需考虑其取向。
Brivio等人\upcite{brivioStructuralElectronicProperties2013}对\ce{MA}的不同取向进行研究,认为$<100>$、$<110>$和$<111>$三种取向中以$<100>$取向热力学最稳定;而Giorgi\upcite{giorgiSmallPhotocarrierEffective2013}等人认为$<111>$最稳定。
实验上,Wasylishen等人\upcite{wasylishenCationRotationMethylammonium1985}通过光谱测量证实了室温下有机阳离子甲胺存在快速旋转;
而Frost等人\upcite{frostAtomisticOriginsHighPerformance2014}计算了立方相\ce{APbI3}(\ce{A}=\ce{NH4},\ce{CH3NH3},\ce{CF3NH3}和\ce{NH2CHNH2})中有机分子的旋转势垒,其势垒大小受有机分子的极性和尺寸影响,最低为\SI{0.3}{kJ \per \mole}(\ce{NH4}), 最高为\SI{13.9}{kJ \per \mole}(\ce{NH2CHNH2})。其中甲胺离子的旋转能垒为\SI{1.3}{kJ \per \mole},即\SI{0.06}{eV}。

钙钛矿中的无机框架由B位金属阳离子和X位卤族元素构成。
通过更换或掺杂同族原子,可以调控材料的带隙、晶格尺寸和稳定性\upcite{liRecentProgressOrganohalide2015}。

此外,由于有机分子的存在,在研究该类钙钛矿材料的性质时需要特别考虑范德华力及其他可能的相互作用。
实验和计算模拟\upcite{wangDensityFunctionalTheory2013, puScreeningPerovskiteMaterials2021, varadwajSignificanceHydrogenBonding2019}上均证实有机金属卤化物钙钛矿中存在氢键及其他非共价相互作用,并对材料的结构、能带等诸方面均有影响。

综上,研究有机金属卤化物钙钛矿,如\ce{MAPbI3},需要特别考虑其中有机阳离子的取向和转动,以及可能存在的氢键等相互作用对输运能垒的影响;此外,合理设计其中的金属阳离子和卤族元素可以提高材料稳定性等性能指标。

\section{计算方法}

随着理论的深入、算力的提高,利用高性能计算对材料的性能进行模拟成为了可能。近年来,理论计算在研究化学反应微观机理、预测材料结构和性能等方面进展突出,很好地揭示了一般实验难以表征的结果,因而引起人们注意。

\subsection{密度泛函理论}

基于密度泛函理论(Density function theory,DFT)的第一性原理计算通过自洽地求解体系电子密度和电子波函数,
可以很好的求解固体系统的稳态结构和电子结构。
因此,许多研究工作将该方法应用于计算模拟材料性能和反应过程,使该方法成为开发高性能锂金属电池材料、揭示反应机理的有效工具\upcite{1review_22}。

DFT的核心为Kohn和Sham\upcite{1review_23}在Thomas\upcite{1review_24}等前人工作上提出的Kohn-Sham方程。
该方程借助体系电子密度,为计算材料体系的稳定构型、总能量和能带结构等提供了理论支撑和计算思路。
下面简要介绍Kohn-Sham 方程的建立过程。

首先,对于无外场作用的多粒子体系,其定态薛定谔方程中哈密顿算符为
\begin{equation}
    \hat{H}=\hat{H}_{\mathrm{e}}+\hat{H}_{\mathrm{N}}+\hat{H}_{\mathrm{eN}}
\end{equation}
其中,$\hat{H}_{\mathrm{e}}$和$\hat{H}_{\mathrm{N}}$分别为电子和原子核各自的能量算符(包括动能$\mathrm{T}$和相互作用能$\mathrm{V}$),
$\hat{H}_{\mathrm{eN}}$为电子与原子核间相互作用的能量算符(对应相互作用能$\mathrm{V}$),即
\begin{gather}
    \hat{H}_e(r) = \hat{T}_e(r)+\hat{V}_{\mathrm{ee}}(r) \\
    \hat{H}_N(R) = \hat{T}_N(R)+\hat{V}_{\mathrm{NN}}(R) \\ 
    \hat{H}_{\mathrm{eN}}(r,R) = \hat{V}_{\mathrm{eN}}(r,R) 
\end{gather}

由于原子核和电子质量上的极大差异,可以两者的运动可以分开考虑:
求解电子运动时,原子核由于质量较大,运动状态较电子不易改变,假设为静止;
求解原子核运动时,不考虑电子的具体运动状态变化,
即Born-Oppenheimer近似\upcite{1review_28}。
于是,在求解体系电子运动时,原子核坐标$R$为常数,体系的能量算符的变量仅为电子坐标$r$。
略去常数,体系的能量算符
\begin{equation}{\label{eq:total}}
    \hat{H}=\hat{T}_{\mathrm{e}}(r)+\hat{V}_{\mathrm{ee}}(r)+\hat{V}_{\mathrm{eN}}(r)
\end{equation}

公式\ref{eq:total}的求解难度随电子数量增多而极速提高,一般而言,无法直接利用公式\ref{eq:total}求解体系的定态薛定谔方程。
Hohenberg和Kohn\upcite{1review_23}将粒子数密度函数$\rho(r)$作为基本变量,并建立与体系的各种性质的关系,即Hohenberg-Kohn定理:不计自旋的全同费米子系统的基态能量是粒子数密度函数$\rho(r)$的唯一泛函;对给定的哈密顿量,能量泛函$E_0[\rho]$对正确的粒子数密度函数$\rho(r)$取极小值,并等于基态能量。其中,$\rho(r)$满足:
\begin{equation}
    \rho(r)=N\int_{}^{}\mathrm{d}r_1 \cdots \int_{}^{}\mathrm{d}r_N | \varPsi (r,r_1,\cdots,r_N) |^2
\end{equation}

由此,对于电子体系能量
\begin{equation}
    E[\rho] = \varPsi ^*(\hat{T}_{\mathrm{e}}+\hat{V}_{\mathrm{ee}}+\hat{V}_{\mathrm{ext}})\varPsi
\end{equation}
其中,需要考虑电子之间的关联、原子核的势场对电子运动的影响,假定所有电子所受局域势影响$\nu(r)$相同,使用$\hat{V}_{\mathrm{ext}}$表示,有
\begin{equation}
    V_{\mathrm{ext}} = \int_{}^{}\mathrm{d}r\nu(r)\rho(r)
\end{equation}

为处理电子间相互作用,首先借助无相互作用电子系统的相关量表示实际存在相互作用系统的参数。
使用$N$个单电子波函数$ \varphi _i(r) $代替密度函数$\rho(r)$,即
\begin{equation}
    \rho(r)=\sum_{i=1}^{N}|\varphi_i (r)|^2
\end{equation}
故,电子动能
\begin{equation}
    T_{\mathrm{e}}=\sum_{i=1}^{N}\int_{}^{}\mathrm{d}r\varphi^*_i(r)(-\frac{1}{2}\nabla^2)\varphi_i(r)
\end{equation}
电子间相互作用
\begin{equation}
    V_{\mathrm{ee}} =\frac{1}{2}\iint_{}^{}\mathrm{d}r\mathrm{d'}\frac{\rho(r)\rho(r')}{|r-r'|}
\end{equation}
最后引入交互关联项$V_{\mathrm{XC}}$,表示上述忽略相互作用所损失的复杂性。

通过变分法,得著名的Kohn-Sham方程\upcite{1review_23}
\begin{equation}
    {-\frac{1}{2}\nabla^2+V_{\mathrm{KS}}[\rho(r)]}\varphi_i(r)=E_i\varphi_i(r)
\end{equation}
其中
\begin{equation}
    V_{\mathrm{KS}}[\rho(r)]=\nu(r)+\int_{}^{}\mathrm{d}r'\frac{\rho(r')}{|r-r'|}+\frac{\delta E_{\mathrm{XC}}[\rho(r)]}{\delta\rho(r)}
\end{equation}

上述Kohn-Sham方程使得求解体系电子结构成为可能,但仍存在未确定项$V_{\mathrm{XC}}$,尚不能确定其具体形式。
对此,人们使用诸多近似方法进行估计,如局部密度近似(LDA)、广义梯度近似(GGA)及其衍生出的PW86\upcite{perdewAccurateSimpleDensity1986}和PW91\upcite{perdewAccurateSimpleAnalytic1992}等。

LDA认为$V_{\mathrm{XC}}$只与局域密度有关,因此通过求解均匀电子气可以获得$V_{\mathrm{XC}}$的一般解析形式;
GGA由Perdew\upcite{1review_28}提出,额外考虑了周围电子密度对$V_{\mathrm{XC}}$的影响,使用电子密度梯度一进步提高$V_{\mathrm{XC}}$的准确性。
因此,LDA对于电子密度变化比较平缓的体系(如金属)计算较为精确,但对多数化合物存在较大偏差;
相比而言,GGA适用于更多体系。

改善$V_{\mathrm{XC}}$偏差的另一个思路是考虑杂化泛函。
将GGA的交换能$E^{GGA}_{\mathrm{X}}$与非局域的Hartree-Fock型交换能$E^{GGA}_{\mathrm{X}}$以一定比例$\alpha$混合,再与GGA的关联能$E^{GGA}_{\mathrm{C}}$共同组成新形式的交换关联能$E_{\mathrm{XC}}$,即
\begin{equation}
    E_{\mathrm{XC}}=\alpha E^{HF}_{\mathrm{X}}+(1-\alpha)E^{GGA}_{\mathrm{X}}+E^{GGA}_{\mathrm{C}}
\end{equation}
进一步将$E^{HF}_{\mathrm{X}}$限制在短程范围,即为HSE型杂化泛函\upcite{heydEfficientHybridDensity2004}。
该泛函因计算带隙更为准确,而被广泛应用。

以上为DFT的主要理论。
在实际应用中,求解过程已通过代码实现,并封装成软件和数据包。
目前用于DFT计算的VASP(Vienna Ab-initio Simulation Package)是泛函数据完备,功能较全面,最流行的商用软件之一。
该软件主要通过INCAR(算法参数)、POSCAR(原子位置)、KPOINTS(能带计算参数)和POTCAR(原子信息)四个输入文件实现体系的结构优化、电子结构计算等功能。

本课题将主要使用VASP软件进行有机金属卤化物体系的计算模拟。

\subsection{微动弹性带方法}

对于化学反应和原子迁移而言,在确定反应物(初始位置)和生成物(最终位置)的结构后,理论上两者间存在能量最小路径(minimum energy path,MEP),而过渡态为其上的最高点,即一阶鞍点。
过渡态与反应物的能量之差即为反应能垒,是研究反应过程的最为基础、重要的参数。
此外,对比过渡态与反应物的结构变化,有助于深入认识反应机理。

微动弹性带(Nudged Elastic Band,NEB)及在此基础上发展的CI-NEB方法,是计算过渡态的常用方法。

首先,在反应物结构和生成物结构之间建立一系列结构(称为 image);之后将相邻的image使用弹簧力进行连接,形成具有弹性的链条(即Elastic Band);进而,对这一链条整体进行优化,使其受力最小,进而得到MEP;最后,结合能量、振动频率等信息,确定过渡态。

而CI-NEB方法则在NEB原方法之上,分解掉原子在特定方向上的受力,提高了过渡态搜索效率。

本课题将主要使用CI-NEB方法进行锂离子迁移过程的模拟和求解。

\section{论文选题依据及安排}

在最近的研究中,Yao等人\upcite{yinMetalChloridePerovskite2020}设计了有机金属卤化物钙钛矿型保护层,实验上证明该新型保护层在抑制枝晶生长、保证电池良好循环性能上有很好的效果。
然而,Yao等人的工作对锂离子在有机金属卤化物钙钛矿型保护层中的输运机理研究不足。
他们的工作中仅就单个锂离子在\ce{MASnCl3}和\ce{MAPbCl3}中的输运路径进行初步计算模拟,在极性分子取向、多输运路径、协同输运等方面并未深入研究。因此,本课题在Yao等人的基础之上,通过计算模拟对有机金属卤化物钙钛矿型锂金属电极保护层锂离子输运的机制进行更为系统深入的研究。

论文安排如下:

\begin{enumerate}
    \item 有机金属卤化物钙钛矿中存在大量氢键等相互作用,这可能对锂离子的输运机理等产生较大影响。因此,论文首先实现对其中氢键等成键的强度估计,并验证该方法的有效性。
    \item 由于有机阳离子具有取向,锂离子在其中存在多个不等价的嵌入位置。论文将此就建立并计算分析锂离子嵌入的稳定模型,包括几何结构、电子结构、成键情况等。
    \item 在锂离子嵌入模型的基础之上,建立并计算锂离子在不同位点间的迁移模型,并对其过渡态进行分析。论文将在最后说明氢键对锂离子输运过程的影响。
\end{enumerate}

