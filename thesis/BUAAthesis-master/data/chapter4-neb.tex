% !Mode:: "TeX:UTF-8"
\chapter{锂离子在有机金属卤化物钙钛矿中迁移的计算}

\section{引言}

在章\ref{ch:pos}中建立并计算了锂离子嵌入模型,
获得锂离子三个不等价位点的稳定结构。
该章将首先基于上述三个结构,利用CI-NEB 方法建立锂离子在点位点间迁移的模型,以模拟锂离子输运过程;
之后针对其中的过渡态结构进行分析,以说明锂离子迁移能垒低的原因。

\section{模型搭建}

\subsection{等价位点的模型结构}

为模拟锂离子输运的全过程,需要使用等价的位点作用为迁移的始末态进行CI-NEB建模。
根据前文对锂离子嵌入位置的分析,不等价位点无法通过对称性重合,而等价位点间仅能通过平移对称性实现重合。
理论上,等价的位点环境相同,锂离子嵌入等价位点的结构优化结果也同样通过平移对称性重合。
但在实际计算中,由于存在计算误差,不能保证平移等价的初始结构在结构优化后依然具有平移等价的性质。

针对上述问题,并为减少计算量,本课题通过算法,利用平移等价的性质,已知结构的平移等价结构。
该方法主要包括两步:
\begin{enumerate}
    \item 根据周期性条件确定原子坐标的平移操作矩阵,对记录体系原子坐标的POSCAR进行平移操作;
    \item 根据理想结构确定平移操作后POSCAR中原子的排列顺序,重新排列原子使平移前后原子一一对应(对于VASP软件,在计算CI-NEB时必须保证整始末状态的原子一一对应)。
\end{enumerate}

\begin{figure}[htbp]
    \centering
    \subfigure{
    \includegraphics[width=0.45\textwidth]{../data/111-121-neb1/00_CONTCAR.png}}
    \subfigure{
    \includegraphics[width=0.45\textwidth]{../data/111-121-neb1/13_CONTCAR.png}}
    \caption{\ce{Li+}嵌入\ce{MASnCl3}pos1位点的结构优化后结构及利用算法生成b方向平移1个单位后的等价结构}
    \label{fig:111-pos1-eq}
\end{figure}


这样,只需对三种不等价位点的锂离子嵌入模型进行结构优化,就可以快速获得所有位点的锂离子嵌入模型的优化结构,如图\ref{fig:111-pos1-eq}。
该方法既保证了始末状态的在平移对称性下等价,又减少了额外的计算代价。

\subsection{锂离子迁移方向}

根据有机金属钙钛矿结构(图\ref{fig:111-pos-contcar}),锂离子嵌入的等价位点间仅能通过平移对称性重合,但是由于等价位点间的连线上存在无机框架种的卤原子,锂离子无法直接在等价点位间进行迁移,需要借助临近的八面体空隙,“翻越”卤原子,迁移至等价位点。
此外,锂离子嵌入不同位点时,甲胺离子会发生不同程度的旋转,这意味模拟锂离子迁移必须考虑甲胺离子的旋转行为。
因此,使用如下方法模拟锂离子的完整迁移过程:
\begin{enumerate}
    \item 使用插值方法,建立并计算相邻不等价位点间锂离子迁移模型(例如,锂离子从pos1迁移至pos2);
    \item 使用平移算法,生成等价位点的结构模型(例如,根据pos1结构向b方向平移1个单位生成等价的pos1-eq结构),建立并计算不等价位点至新位点间锂离子迁移模型(例如,锂离子从pos2迁移至pos1-eq);
    \item 将上述两个过程相加,得完整路径,进一步提高精度进行计算(例如,锂离子从pos1首先迁移至pos2,再迁移至pos1-eq)。
\end{enumerate}

% TODO:插入示意图

体系总共存在三个不等价的位点pos1、pos2和pos3,这意味着存在多条不等价的锂离子迁移路径。
本课题针对a、b和c三个晶格矢量方向,各选择一条锂离子迁移路径进行计算模拟。
并且,考虑到锂离子靠近胺基时会破坏原本较强的氢键,使体系能量明显升高,
因此,在选择锂离子迁移路径时,优先选择远离胺基的路径。


综上,对\ce{MASnCl3}、\ce{MASnBr3}、\ce{MASnI3}和 \ce{MAPbI3}四种体系建立的模型如下:
\begin{enumerate}
    \item pos1-eq:pos1结构向b方向平移1个单位的等价结构;
    \item pos1-eqeq:pos1结构向-a方向平移1个单位的等价结构;
    \item pos2-eq:pos2结构向-c方向平移1个单位的等价结构;
    \item NEB121:\ce{Li+} 由pos1位点迁移至pos2位点,再迁移至pos1-eq位点,沿b方向迁移;
    \item NEB131:\ce{Li+} 由pos1位点迁移至pos3位点,再迁移至pos1-eqeq位点,沿-a方向迁移;
    \item NEB212:\ce{Li+} 由pos2位点迁移至pos1位点,再迁移至pos2-eq位点,沿-c方向迁移。
\end{enumerate}

\begin{figure}[htbp]
    \centering
    \subfigure{
    \includegraphics[width=0.3\textwidth]{../data/111-121-neb1/Li_path_CONTCAR.png}
    }
    \subfigure{
    \includegraphics[width=0.3\textwidth]{../data/111-131-neb2/Li_path_CONTCAR.png}
    }
    \subfigure{
    \includegraphics[width=0.3\textwidth]{../data/111-212n-neb3/Li_path_CONTCAR.png}
    }
    \caption{NEB121、NEB131、NEB212迁移模型的示意图}
    \label{fig:111-neb-Li-path}
\end{figure}


\section{计算结果分析}

\subsection{计算参数}

该部分计算主要为CI-NEB计算,较之前结构优化计算需额外设置控制NEB计算的参数,以及合适的收敛判据。

控制NEB计算的参数主要有:
IMAGES,NEB算法中过程image的数量,对于子过程设置为6或7;SPRING,连接相邻image的弹簧力;LCINEB=.TRUE.,ICHAIN=0,IBRION=1,POTIM=0,IOPT=1,开启CINEB计算。
收敛判据方面,考虑到计算体系庞复杂,为提高效率,使用两个级别计算精度,在粗收敛后进一步精收敛。
粗收敛设置为EDIFF=1E-5,EDIFFG=-0.05;
精收敛设置为EDIFF=1E-7,EDIFFG=-0.01。

其余参数与结构优化计算时相同。

\subsection{迁移过程}

\ce{MASnCl3}、\ce{MASnBr3}、\ce{MASnI3}和 \ce{MAPbI3}四种体系的计算结果存在一定差异,
但限于篇幅这里仅展示\ce{MASnCl3}中三条迁移路径的计算结果,其余结果见附录\ref{ap:nebpos}。

\begin{figure}[htbp]
    \centering
    \subfigure[00]{
    \includegraphics[width=0.3\textwidth]{../data/111-121-neb1/00_CONTCAR.png}
    }
    \subfigure[03]{
    \includegraphics[width=0.3\textwidth]{../data/111-121-neb1/03_CONTCAR.png}
    }
    \subfigure[04]{
    \includegraphics[width=0.3\textwidth]{../data/111-121-neb1/04_CONTCAR.png}
    } \\
    \subfigure[05]{
    \includegraphics[width=0.3\textwidth]{../data/111-121-neb1/05_CONTCAR.png}
    }
    \subfigure[08]{
    \includegraphics[width=0.3\textwidth]{../data/111-121-neb1/08_CONTCAR.png}
    }
    \subfigure[10]{
    \includegraphics[width=0.3\textwidth]{../data/111-121-neb1/10_CONTCAR.png}
    } \\ 
    \subfigure[13]{
    \includegraphics[width=0.3\textwidth]{../data/111-121-neb1/13_CONTCAR.png}
    }
    \caption{\ce{MASnCl3}的NEB121模型的部分image结构计算结果(image=00,03,04,05,08,10,13)}
    \label{fig:111-neb121-contcar}
\end{figure}

\begin{figure}[htbp]
    \centering
    \subfigure[00]{
    \includegraphics[width=0.3\textwidth]{../data/111-131-neb2/00_CONTCAR.png}
    }
    \subfigure[04]{
    \includegraphics[width=0.3\textwidth]{../data/111-131-neb2/04_CONTCAR.png}
    }
    \subfigure[05]{
    \includegraphics[width=0.3\textwidth]{../data/111-131-neb2/05_CONTCAR.png}
    } \\
    \subfigure[07]{
    \includegraphics[width=0.3\textwidth]{../data/111-131-neb2/07_CONTCAR.png}
    }
    \subfigure[10]{
    \includegraphics[width=0.3\textwidth]{../data/111-131-neb2/10_CONTCAR.png}
    }
    \subfigure[11]{
    \includegraphics[width=0.3\textwidth]{../data/111-131-neb2/11_CONTCAR.png}
    } \\ 
    \subfigure[13]{
    \includegraphics[width=0.3\textwidth]{../data/111-131-neb2/13_CONTCAR.png}
    }
    \caption{\ce{MASnCl3}的NEB131模型的部分image结构计算结果(image=00,04,05,07,10,11,13)}
    \label{fig:111-neb131-contcar}
\end{figure}

\begin{figure}[htbp]
    \centering
    \subfigure[00]{
    \includegraphics[width=0.3\textwidth]{../data/111-212n-neb3/00_CONTCAR.png}
    }
    \subfigure[03]{
    \includegraphics[width=0.3\textwidth]{../data/111-212n-neb3/03_CONTCAR.png}
    }
    \subfigure[05]{
    \includegraphics[width=0.3\textwidth]{../data/111-212n-neb3/05_CONTCAR.png}
    } \\
    \subfigure[08]{
    \includegraphics[width=0.3\textwidth]{../data/111-212n-neb3/08_CONTCAR.png}
    }
    \subfigure[11]{
    \includegraphics[width=0.3\textwidth]{../data/111-212n-neb3/11_CONTCAR.png}
    }
    \subfigure[12]{
    \includegraphics[width=0.3\textwidth]{../data/111-212n-neb3/12_CONTCAR.png}
    } \\ 
    \subfigure[14]{
    \includegraphics[width=0.3\textwidth]{../data/111-212n-neb3/14_CONTCAR.png}
    }
    \subfigure[16]{
    \includegraphics[width=0.3\textwidth]{../data/111-212n-neb3/16_CONTCAR.png}
    }
    \caption{\ce{MASnCl3}的NEB212模型的部分image结构计算结果(image=00,03,05,08,11,12,14,16)}
    \label{fig:111-neb212-contcar}
\end{figure}

图\ref{fig:111-neb121-contcar}、图\ref{fig:111-neb131-contcar}和图\ref{fig:111-neb212-contcar}分别为\ce{Li+}嵌入\ce{MASnCl3}的NEB模型的计算结果中部分image结构。

可以直观看出,锂离子迁移过程中伴随着甲胺离子的转动,锂离子始终接近甲基,而远离胺基。
该现象可用章节\ref{ch:pos}的结论进行解释:
甲胺离子不同取向间的能量相近,且能垒较低(室温下,甲胺离子会发生快速旋转\upcite{wasylishenCationRotationMethylammonium1985},旋转能垒约为 \SI{0.05}{eV}\upcite{frostAtomisticOriginsHighPerformance2014});
由于氢键强度不同,锂离子嵌入甲基与卤原子之间时,其位阻远低于胺基;
因此,当锂离子在不同位点间迁移时,甲胺离子会以较小的能垒旋转,使胺基端远离锂离子,从而减少锂离子嵌入导致的位阻。

锂离子迁移也对无机框架产生影响,但框架依然维持基本结构。
由于电负性相差较大,锂离子与无机框架中的卤原子形成离子键。
根据图\ref{fig:111-neb121-contcar}、图\ref{fig:111-neb131-contcar}和图\ref{fig:111-neb212-contcar},
迁移过程中,锂离子与附近3至5个卤原子相互作用。
静电力作用使卤原子向锂离子靠近,由此导致无机八面体发生旋转和畸变。

\begin{figure}[htbp]
    \centering
    \includegraphics[width=0.75\textwidth]{contrast-P-111-neb-barrier.png}
    \caption{\ce{Li+}在\ce{MASnCl3}中迁移时体系能量变化}
    \label{fig:111-barrier}
\end{figure}

迁移过程中的能量变化值得特别注意,其中的迁移能垒(energy barrier)反映了离子输运的难度。
对于锂金属电极保护膜而言,低能垒意味着更高的离子迁移数,保证了锂电池的工作性能。
图\ref{fig:111-barrier}即为\ce{MASnCl3}体系下 \ce{Li+}迁移导致的体系能量变化。
首先,由图可以直接获得迁移能垒:NEB121、NEB131和NEB212的能垒分别为 \SI{0.29}{eV}、\SI{0.51}{eV}和 \SI{0.32}{eV}。
该结果与Yao等人\upcite{yinMetalChloridePerovskite2020}的计算结果 \SI{0.45}{eV} 基本一致,但本课题使用更精确的计算方法,结果更为可靠。
该迁移能垒与锂离子电池中固态电解质的迁移能垒相近,这意味着有机金属钙钛矿属于快锂离子导体。
其次,不同方向的迁移能垒存在明显差异,即具有各向异性。
该现象没有得到Yao等人\upcite{yinMetalChloridePerovskite2020}的注意。
各向异性的产生来源于无机框架中的甲胺离子取向。
理想结构中,甲胺离子的碳氮键与晶格的b、c方向相近,与a方向垂直;
相应的,锂离子在b方向(NEB121)、c方向(NEB212)的迁移能垒低于a方向(NEB131),
即锂离子迁移方向与甲胺离子取向相近时,迁移能垒相对较低;与取向垂直时,迁移能垒相对较高。
该结论同样适用于\ce{MASnBr3}、\ce{MASnI3}体系(图\ref{fig:112-barrier}和图\ref{fig:113-barrier})。

\begin{figure}[htbp]
    \centering
    \includegraphics[width=0.75\textwidth]{112-all-barrier.png}
    \caption{\ce{Li+}在\ce{MASnBr3}中迁移时体系能量变化}
    \label{fig:112-barrier}
\end{figure}

\begin{figure}[htbp]
    \centering
    \includegraphics[width=0.75\textwidth]{113-all-barrier.png}
    \caption{\ce{Li+}在\ce{MASnI3}中迁移时体系能量变化}
    \label{fig:113-barrier}
\end{figure}

为揭示上述各向异性的产生原因,有必要先对迁移过程中的能量变化进行分析。
根据计算模拟结果,可以将锂离子迁移过程分解为两部分:甲胺离子的转动和锂离子的移动。
锂离子的嵌入对甲胺离子产生位阻,促使甲胺离子转动以降低整体能量;
甲胺离子的转动又为锂离子的移动提供空间。
由于甲胺离子相对庞大,并且锂离子迁移过程中体系的8个甲胺离子都会发生不同程度的旋转,因此甲胺锂离子的转动在完整的迁移过程中占有很大比例。
根据章节\ref{ch:pos}甲胺锂离子的取向对体系能量的影响主要源于甲胺离子与无机框架中卤原子所成氢键的强弱。
因此,对迁移过程中的所有氢键键能变化进行统计。
为提高准确性,将甲基端对卤原子所成的微弱氢键同样加入考虑。
根据章节 \ref{ch:bcp}的结论,氢键的键能与bcp处电子密度成正比关系,但存在正向偏差,无能定量计算。
所以,仅对迁移过程中的所有氢键bcp处电子密度进行统计,半定量地表示氢键键能变化。

\begin{figure}[htbp]
    \centering
    \subfigure{
    \includegraphics[width=0.3\textwidth]{111-121-barrier.png}
    }
    \subfigure{
    \includegraphics[width=0.3\textwidth]{111-131-barrier.png}
    }
    \subfigure{
    \includegraphics[width=0.3\textwidth]{111-212-barrier.png}
    } \\ 
    \subfigure{
    \includegraphics[width=0.3\textwidth]{111-121-H-bcp-sum.png}
    }
    \subfigure{
    \includegraphics[width=0.3\textwidth]{111-131-H-bcp-sum.png}
    }
    \subfigure{
    \includegraphics[width=0.3\textwidth]{111-212-H-bcp-sum.png}
    } 
    \caption{\ce{MASnCl3}体系中NEB121、NEB131和NEB212模型能量变化和氢键bcp处电子密度总和变化}
    \label{fig:111-barrier-bcp}
\end{figure}

总能量与氢键bcp处电子密度总和的变化如图\ref{fig:111-barrier-bcp}所示。
氢键bcp处电子密度总和半定量的反映了体系氢键总键能。
NEB121、NEB131的能量与氢键总键能的变化趋势向相反,极值点十分吻合。
当氢键总键能升高时,意味着体系形成数量更多、强度更高的氢键,相应的体系总能量下降。
相比而言,NEB212的结果吻合度并不高,氢键键能变化幅度更小。
对比NEB121、NEB131和NEB212的结构可以发现,NEB212的无机八面体旋转得更为明显。
由此推测NEB212在计算模拟时进行了“过度优化”,通过调整无机框架,使体系在甲胺离子旋转时依然形成较高强度的氢键。

对于其余的\ce{MASnBr3}、\ce{MASnI3}和 \ce{MAPbI3}三种体系,上述结论同样适用。

综上,甲胺离子旋转导致的氢键断裂和形成,以及由此引起的无机框架畸变,影响了锂离子迁移过程中能量变化。

\subsection{过渡态结构}

计算模拟初步表明锂离子在\ce{MASnCl3}等体系下具有较低的迁移能垒,该现象与甲胺离子形成的氢键有关。
为深入说明该原因,有必要针对迁移过程中的过渡态结构进行分析,通过与初始结构对比,确定低能能垒的原因。

针对\ce{MASnCl3}体系下NEB121模型,选择其中的过渡态图像03、05和10进行体系结构和电子结构的分析。

\begin{figure}[htbp]
    \centering
    \subfigure{
    \includegraphics[width=0.3\textwidth]{../data/111-121-neb1/03/CONTCAR.png}
    }
    \subfigure{
    \includegraphics[width=0.3\textwidth]{../data/111-121-neb1/05/CONTCAR.png}
    }
    \subfigure{
    \includegraphics[width=0.3\textwidth]{../data/111-121-neb1/10/CONTCAR.png}
    } \\ 
    \subfigure{
    \includegraphics[width=0.3\textwidth]{../data/111-121-neb1/03/CHGDIFF.png}
    }
    \subfigure{
    \includegraphics[width=0.3\textwidth]{../data/111-121-neb1/05/CHGDIFF.png}
    }
    \subfigure{
    \includegraphics[width=0.3\textwidth]{../data/111-121-neb1/10/CHGDIFF.png}
    } 
    \caption{\ce{MASnCl3}NEB121模型中图像03、05和10的体系结构和电荷密度差分}
    \label{fig:111-121-trans-contcar-diff}
\end{figure}

\begin{figure}[htbp]
    \centering
    \subfigure{
    \includegraphics[width=0.3\textwidth]{../data/111-121-neb1/03/CHGDIFF-plan.png}
    }
    \subfigure{
    \includegraphics[width=0.3\textwidth]{../data/111-121-neb1/05/CHGDIFF-plan.png}
    }
    \subfigure{
    \includegraphics[width=0.3\textwidth]{../data/111-121-neb1/10/CHGDIFF-plan.png}
    } \\ 
    \subfigure{
    \includegraphics[width=0.3\textwidth]{../data/111-121-neb1/03/CHGDIFF-2D.png}
    }
    \subfigure{
    \includegraphics[width=0.3\textwidth]{../data/111-121-neb1/05/CHGDIFF-2D.png}
    }
    \subfigure{
    \includegraphics[width=0.3\textwidth]{../data/111-121-neb1/10/CHGDIFF-2D.png}
    } 
    \caption{\ce{MASnCl3}NEB121模型中图像03、05和10的电荷密度差分截面图}
    \label{fig:111-121-trans-diff-plan}
\end{figure}

图\ref{fig:111-121-trans-contcar-diff}为过渡态03、05和10的体系结构和电荷密度差分。
由图可知,锂离子嵌入导致的空间电荷密度变化,在形式上与pos1等嵌入点结构相似,
均表现为锂离子与临近卤原子间电荷密度明显增多,锂离子与甲胺离子的甲基之间电荷密度少量增多。

\begin{table}[htbp]
    \begin{center}
        \caption{锂离子与其临近原子(团)的整体电荷数(\ce{Li}1号、\ce{H}14号、\ce{C}6号、\ce{CH3}6号、\ce{NH3}6号、\ce{CH3NH3}6号、\ce{Cl}6号和\ce{Sn}2号,单位\si{e})}
        \begin{tabular}{ccccccccc}
            \toprule
             & \ce{Li} & \ce{H} & \ce{C} & \ce{CH3} & \ce{NH3} & \ce{CH3NH3} & \ce{Cl} & \ce{Sn} \\
            \midrule
            理想结构 & - & +0.073 & +0.198 & +0.482 & +0.327 & +0.808 & -0.705 & +1.272\\
            无 \ce{Li+}03结构 & (+1.000) & +0.099 & +0.193 & +0.484 & +0.318 & +0.802 & -0.694 & +1.252 \\
            \ce{Li+}嵌入03结构 & +0.894 & +0.077 & +0.137 & +0.453 & +0.335 & +0.788 & -0.730 & +1.271 \\
            \ce{Li+}嵌入pos1结构 & +0.901 & +0.082 & +0.186 & +0.439 & +0.341 & +0.780 & -0.724 & +1.287\\
            \bottomrule
        \end{tabular}
        \label{tb:trans-bader}
    \end{center}
\end{table}

针对过渡态03进行bader电荷计算,并与初始结构pos1的bader电荷计算结果相比较。
根据表\ref{tb:trans-bader}可知,过渡态03较初始态pos1的bader电荷计算结果仅存在微小差异。
这意味着锂离子迁移过程中体系原子间并未发生明显的电荷转移。

综上,可以确定,影响了锂离子迁移过程中能量变化的主要因素为甲胺离子与卤原子间的氢键,以及锂离子与无机框架中卤原子所成离子键。

\section{结论}

该部分分别建立了\ce{MASnCl3}、\ce{MASnBr3}、\ce{MASnI3}和 \ce{MAPbI3}四种体系的三种锂离子迁移模型,并对计算获得的过渡态结构进行电子自洽计算。

计算结果表明,
\begin{enumerate}
    \item 锂离子在有机金属卤化物钙钛矿中具有较低的迁移能垒(\SI{0.2}{eV}至\SI{0.5}{eV}),且能垒具有各向异性,在与甲胺离子原本取向相近的方向上能垒相对较低;
    \item 甲胺离子随锂离子迁移而发生旋转,以较低的能量代价减小锂离子位阻,该旋转机制是迁移能垒较低和各向异性的原因;
    \item 锂离子在迁移过程中,与周围3至5个卤原子形成离子键,导致无机八面体发生畸变,但无机框架依然维持稳定。 
\end{enumerate}

