% !Mode:: "TeX:UTF-8"

% 学院中英文名,中文不需要“学院”二字
% 院系英文名可从以下导航页面进入各个学院的主页查看
% https://www.buaa.edu.cn/jgsz/yxsz.htm
\school
{材料科学与工程}{School of Materials science and Engineering}

% 专业中英文名
\major
{材料科学与工程}{Materials science and Engineering}

% 论文中英文标题
\thesistitle
{有机金属卤化物钙钛矿型锂金属电极保护层锂离子输运机制研究}
{} % 副标题
{Organic metal halide perovskite type lithium metal electrode layer lithium ion transport mechanism research}
{}

% 作者中英文名
\thesisauthor
{华广斌}{Hua Guangbin}

% 导师中英文名
\teacher
{张千帆}{Zhang Qianfan}
% 副导师中英文名
% 注:慎用‘副导师’,见北航研究生毕业论文规范
%\subteacher{(副导师姓名)}{(Name of Subteacher)}

% 中图分类号,可在 http://www.ztflh.com/ 查询
\category{O484.3}

% 本科生为毕设开始时间;研究生为学习开始时间
\thesisbegin{2022}{3}{1}

% 本科生为毕设结束时间;研究生为学习结束时间
\thesisend{2022}{5}{26}

% 毕设答辩时间
\defense{2022}{6}{2}

% 中文摘要关键字
\ckeyword{有机金属卤化物钙钛矿,锂离子迁移,计算模拟,氢键}

% 英文摘要关键字
\ekeyword{Organic metal halide perovskite, Lithium ion migration, Computation and Simulation, Hydrogen Bond}
