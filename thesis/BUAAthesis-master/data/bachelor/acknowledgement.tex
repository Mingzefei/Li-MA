% !Mode:: "TeX:UTF-8"
\chapter*{致谢}
\addcontentsline{toc}{chapter}{致谢}

在正式致谢前,需要交代一些背景。
这篇论文的作者于2018年9月进入北航,在航空航天大类培养一学年后,自认为数理基础良好,立志于从事儿时向往的科研,因此进入北航材料学院学习。
但是作者在课堂上很快感到不安,并试图自行解决。
之后两年,作者参考理科生培养方案,补充相应课程;在大二下学期向张千帆老师申请进入实验室科研,学习第一性原理计算,并在2020年底正式开始当前课题。
然而,不安感并没有消除,作者转向学长和师兄前辈寻找解决办法。
最终在大三下学期确定研究生去向的前夕,作者决定离开材料专业,并在最后一学年中参与实习,补习新专业课程。
在本科生涯尾声,作者将科研结果匆匆写为这篇毕业论文。

首先,我要向张千帆老师献上真挚的感谢!
不论是学术上认真细致的指导,还是生活上无微不至的关怀,都令我受益颇多,内心感激。
初学计算理论时,提醒我分清主次;
开始进行课题时,指导我高效的实验方法;
遇到困难情绪低落时,鼓励我振奋精神。
我在后期转换专业,参与实习,课题停滞许久,但您对此包容,依然热心指导我推进实验。
在课题组的两年时间里,我完成人生中第一篇科研论文,有了自己的课题,目前正撰写当前课题成果的论文。
这份宝贵的经历,让我认识材料科研,认真审视未来规划。
张老师,感谢您这两年的指导与关爱!

我要感谢实验室的王天帅师兄、刘晓鹏学长、曲家乐师兄、樊晏辰师兄和冯翔师兄。
感谢你们对我不厌其烦的解答和指导,让我能够及时发现问题,快速上手科研;
感谢你们在我怀疑忧虑时,给予我宝贵的建议和帮助。
这里,祝各位师兄前程似锦!

我要感谢我的父母。
你们的开明,你们对我始终不变的信任和支持,让我更有信心和勇气坚持并实践自己的想法。
你们始终以有我这样的儿子而骄傲自豪,我也始终以有你们这样的父母而幸福。
真诚祝愿父母健康长寿!

我还要感谢本科阶段一起参加比赛,合作项目的室友、同学和朋友。
你们横溢的才华、开阔的视野、乐观的心态、坚毅的品格,让我倍感幸与你们结识、同行。

最后,再次感谢所有给予我帮助的人!

\cleardoublepage
