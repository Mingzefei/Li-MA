% !Mode:: "TeX:UTF-8"
% 任务书中的信息
%% 原始资料及设计要求
\assignReq
{模拟锂离子嵌入氯锡甲胺类钙钛矿}
{模拟锂离子在氯锡甲胺类钙钛矿中迁移}
{揭示锂离子在氯锡甲胺类钙钛矿中迁移机制}
{}
{}
%% 工作内容
\assignWork
{实现并验证了适用于VASP软件计算结果的氢键强度估计方法}
{建立并计算锂离子嵌入模型,阐明相关现象}
{建立并计算锂离子迁移模型,揭示相关机制}
{}
{}
{}
%% 参考文献
\assignRef
{华广斌, 樊晏辰, 张千帆. 计算模拟在锂金属负极研究中的应用[J]. 物理化学学报, 2021, 37(2): 2008089.}
{Yin Yi-Chen, Wang Qian, Yang Jing-Tian, et al. Metal chloride perovskite thin film based interfacial
layer for shielding lithium metal from liquid electrolyte[J]. Nature Communications,
2020, 11(1):1761.}
{Emamian Saeedreza, Lu Tian, Kruse Holger, et al. Exploring Nature and Predicting Strength of Hydrogen Bonds: A Correlation Analysis Between Atoms-in-Molecules Descriptors, Binding Energies, and Energy Components of Symmetry-Adapted Perturbation Theory[J].
Journal of Computational Chemistry, 2019, 40(32):2868-2881.}
{}
{}
{}
{}
{}
